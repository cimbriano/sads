\documentclass[11pt, letterpaper, oneside]{article}
\usepackage{enumerate}
\usepackage{ calc }
\usepackage{ amssymb }

\begin{document}

\title{CMCS 818J Project Milestone Report}
\author{Behzad Koosha, Christopher Imbriano}

\maketitle

\section{Abstract}

	A scheme for achieving a streaming authenticated data structure (SADS) was proposed by Papamanthou and Shi \cite{sads}.
	The goal of this project is twofold.
	First, we will implement an example of this data structure and second, we will build an application using the SADS primitive.


\section{Introduction}

	The introduced scheme, streaming authenticated data structure, does not require any interaction between the client and the server while the stream is observed.
	Three important properties are obtained using this scheme :
	\begin{enumerate}[a.]
	\item Independence of prover and verifier : the prover and the verifier update their status independently and with no interaction.
	\item Efficiency : It can be shown that an exponential improvement for many commonly queries in the prover's running time can be achieved.
	\item Expressiveness : a wide range of queries such as membership/ non-membership, successor, range search and frequencies can be supported using this scheme.
	\end{enumerate}

\section{Our Approach}

	The description of the SADS scheme reads much like a protocol, focusing on the behaviour of each of the prover and verifier during a streaming phase and a query phase.
	Before tackling the problem at the protocol level however, we must first implement the basic data structures each of the verifier and prover will need to perform their roles.
	This will be our starting point.

	The road map for this project is as follows:
	\begin{enumerate}
	\item Implement underlying Merkle tree with homomorphic hash function
	\item Implement the SADS scheme
	\item Develop proof-of-concept application using the SADS scheme implementation
	\end{enumerate}

	\subsection{Underlying Data Structures}

	A crucial component of the SADS algorithm relies on matrix multiplication, a relatively expensive computational operation.
	When formulating our plan of action for this project, we were recommended a few linear algebra libraries, which will be discussed in a later section.
	While we recognize the importance of fast matrix operations and parallelism for production quality code, we thought that the added complexity, as well as unknowns with respect to implementation details, would be prohibitive.
	Instead, we've decided to use an interpreted language and the simple matrix operations that come with the standard libraries to start.

	Our language of choice for this first stage of the project is Ruby. Ruby has a number of advantages including:

	\begin{enumerate}
	\item Built-in support of matrix and vector operations
	\item Vibrant developer community that supports both new and experienced developers through learning resources, tool development, etc...
	\item Ability to write compiled extensions in C
	\item Numeric and visualization library, SciRuby
	\end{enumerate}

	\subsection{SADS Scheme}

	Once the underlying data structure is complete, the SADS scheme will use it as a primitive.
	Though one of the scheme's strengths is the independence of the prover and verifier during the streaming setting, the two parties will necessarily need to communicate during the query phase.
	Additionally, there may be communication between the two parties to initiate the streaming setting.
	For these reasons, we anticipate creating an application which will require network communication as it will likely be the case that the verifier and prover are not collocated.

	We have a few frameworks in mind for this task.
	Two of our options at the moment are EventMachine and Node.js.
	Node.js is a "platform for building fast, scalable network applications" and has been gaining popularity in the Ruby community. http://nodejs.org/
	EventMachine is a "fast, simple event-processing library for Ruby programs."http://rubyeventmachine.com/


% In our work, we have generated a secret and public key depending on the size of the stream and the security parameter.
% The next step was to generate the hash function according to each left and right leaf.
% The hash function is computed as $ h(x,y) = Lx + Ry $ where \textbf{L, R} are selected randomly from $Z_{q}^{k \times m}$ and
% \textbf{x,y} are vectors of small norm.

%genkey procedure generates a secret and public key.  The public key includes vectors \textbf{L} and \textbf{R} as well as $q$ and $\mathcal{U}$.


\section{Looking Forward}

	\subsection{Production Quality Libraries}
	It is yet to be seen whether the existing Ruby matrix libraries, either the built in or SciRuby, will be sufficiently efficient for the needs of the SADS scheme.
	That's not to say the current choices of technologies will not produce production quality code but, as mentioned earlier, we have been investigating some optimized libraries listed here.

	\begin{enumerate}
	\item \textbf{NTL 5.5.2} : A free software written in C++ providing data structures and algorithms for arbitrary length integers, for vectors, matrices and polynomials over the integers and over the finite fields and for arbitrary precision floating point arithmetic. \texttt{http://shoup.net/ntl/}
	\item \textbf{MAGMA V2.18} : The kernel of Magma contains implementations of many of the important concrete classes of structure in five fundamental branches of algebra, namely group theory, ring theory, field theory, module theory and the theory of algebras.
	In addition, certain families of structures from algebraic geometry and finite incidence geometry are included. \texttt{http://magma.maths.usyd.edu.au/}
	\end{enumerate}

	\subsection{Semester Outlook}
		While the ultimate goal of the project is to produce a proof-of-concept application which utilizes a SADS scheme, it is best to set manage expectations.
		Our baseline target for this semester is to complete the first two layers described above, that is the underlying data structure and the SADS scheme primitive.
		Having completed these two components, we can then establish time line during which the proof-of-concept application can be hashed out and developed.
		We hope to leverage the Ruby and broader open-source community by packaging this application, or just the SADS scheme primitive, into a library for others to use and provide feedback.


%%%%%%%%%%%%%%%%%%%%%%%%%%%%%%%%%%%%%%%%%%%%%%%%%%%%%%%%%%%%%%%%%%%%%%%%

\begin{thebibliography}{9}
	\bibitem{sads} Elaine Shi, Charalampos Papamanthou \emph{Streaming Authenticated Data Structures}, Date Unknown
\end{thebibliography}

\end{document}

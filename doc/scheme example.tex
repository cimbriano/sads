\documentclass[11pt, letterpaper, oneside]{article}
\usepackage{enumitem}
\usepackage{calc}
\usepackage{ amssymb }

\begin{document}
\section{Step 1: Generate Keys}

	genkey procedure generates a secret and public key.  The public key includes vectors \textbf{L} and \textbf{R} as well as $q$ and $\mathcal{U}$.
	
	Questions on this step:
	\begin{enumerate}
	\item	 In the first algorithm defined on page 10, what does the character in 'sk =  \O' represent? Is that the null set?
	\end{enumerate}

\section{Step 2: Initialization}

	initialize takes the secret key and the public key generated in step 1, and an empty dictionary over the universe of elements and produces
	
	\begin{enumerate}
	\item a table \textbf{T} with N entries (0 to N-1) to represent the Merkle tree leaves
	\item initial state $d_{0} = \textbf{0} \in Z_{q}^{k}$ representing the root digest
	\item a label for each internal node: $\lambda(v)$
	\end{enumerate}
	
	
	Questions on this step:
	\begin{enumerate}
	\item What are the dimensions of table \textbf{T}?  ( $N \times ?$ )
	\item What does q mean in the context of $Z_{q}^{k}$?  Is this a set of $k \times 1$ vectors or a set of $q \times k$ matrices?
	\end{enumerate}
	
\end{document}
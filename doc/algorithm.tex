\documentclass[11pt, letterpaper, oneside]{article}
\usepackage{enumerate}
\usepackage{calc}
\usepackage{ amssymb }

\begin{document}


Questions on this step:
\begin{enumerate}
\item	 In the first algorithm defined on page 10, what does the character in 'sk =  \O' represent? Is that the null set?
\end{enumerate}

\subsection{ 2: Initialization}
\label{initialize}

initialize takes the secret key and the public key generated in step 1, and an empty dictionary over the universe of elements and produces

\begin{enumerate}
\item a table \textbf{T} with N entries (0 to N-1) to represent the Merkle tree leaves
\item initial state $d_{0} = \textbf{0} \in Z_{q}^{k}$ representing the root digest
\item a label for each internal node: $\lambda(v)$
\end{enumerate}


Questions on this step:
\begin{enumerate}
\item What are the dimensions of table \textbf{T}?  ( $N \times ?$ )
\item What does q mean in the context of $Z_{q}^{k}$?  Is this a set of $k \times 1$ vectors or a set of $q \times k$ matrices?
\end{enumerate}


\subsection{ 3: updateVerifier}
\subsubsection{Inputs}
\begin{enumerate}
\item $x$ where $x \in \mathcal{U}$ is the current element of the stream
\item $d_{h}$ current state of $auth(D_{h})$
\item sk and pk - secret and public key \\
\end{enumerate}

\subsubsection{Outputs}
\begin{enumerate}
\item $d_{h + 1}$ updated state of $auth(D_{h + 1})$
\end{enumerate}

\begin{equation}
d_{h+1} = d_{h} + \mathcal{D}_{r}(x, \textbf{1})
\end{equation}

\subsubsection{Example}

\begin{itemize}
\item $\mathcal{U} = \{ 0, 1, 2, 3, 4, 5, 6, 7\}$ similar to example in figure 7.1
\item Suppose $x = 3$
\end{itemize}

Calculating $d_{h + 1}$ reduces to calculating 	$ \mathcal{D}_{r}(x, \textbf{1})$ because $d_{h}$ is known.  \\

In our case, we'd like to calculate $\mathcal{D}_{r}(3, \textbf{1})$, i.e. the digest of the root node with respect 3. \\

By definition 10,
\begin{equation}
\mathcal{D}_{r}(3, \textbf{1}) = \textbf{L} \cdot \textbf{b}(\mathcal{D}_{v_{11}}(3, \textbf{x}_{3}))
\end{equation}
where $v_{11}$ is the left child of $r$, and $\mathcal{D}_{v_{11}}$ is the partial digest of $v_{11}$.

This procedure continues for each internal node eventually reaching the leaf node. The partial digest of a leaf node, in our case, that representing the element $x = 3$ is given by:

\begin{equation}
\mathcal{D}_{3}(3, \textbf{1}) = \textbf{1}
\end{equation}

Substituting the calculated partial digests for each node in the path from the root to the updated leaf node results in:

\begin{equation}
\mathcal{D}_{r}(3, \textbf{1}) = \textbf{L} \cdot \textbf{b} ( \textbf{R} \cdot \textbf{b} ( \textbf{R} \cdot \textbf{b} ( \textbf{1} ) )  )
\end{equation}

Thus, the new digest for the entire tree is given by:
\begin{equation}
d_{h+1} = d_{h} + \textbf{L} \cdot \textbf{b} ( \textbf{R} \cdot \textbf{b} ( \textbf{R} \cdot \textbf{b} ( \textbf{1} ) )  )
\end{equation}

%\subsection{Questions}
%\begin{enumerate}
%\item What are the roles of the secret and public key as inputs?  In the previous email, you said the prover generates the secret key and does not output it. Therefore, the verifier can't have any value here for $sk$ besides null.
%\end{enumerate}

\subsection{ 4: updateProver}
\subsubsection{Input}
\begin{itemize}
\item $x$: $x \in \mathcal{U}$ is the current element of the stream
\item $pk$:  public key
\item $D_{h}$: The table, \textbf{T} which represents leaves of the Merkle tree described by the scheme
\item $auth(D_{h})$: The auxiliary information allowing authentication of the elements in $D_{h}$ including all labels $\lambda(v_{i})$ for internals nodes and $d_{h}$, the root digest of the whole tree.
\end{itemize}
\subsubsection{Output}

\begin{itemize}
\item $D_{h + 1}$: Updated table \textbf{T}
\item $auth(D_{h + 1})$: Updated labels and root digest
\end{itemize}

\subsubsection{Example}

Using the same scenario as the verifier example where $x = 3$, the updated labels is given by:

\begin{equation}
\lambda(v_{i}) = \lambda(v_{i}) + \textbf{b}( \mathcal{D}_{v_{i}}(3, \textbf{1}) ) \textrm{ for } i = \ell, \ell -1, \ldots, 1
\end{equation}

Calculating each $\lambda(v_{i})$ is straight forward.  In the last example, we understood how to calculate a partial digest, and the $\textbf{b}()$ function is well-defined.

\subsection{Questions}

From the algorithm $updateProver$ on page 10,

\begin{quote}
The algorithm sets $\textbf{T}[x]$ = $\textbf{T}[x] + \textbf{1}$, outputting the updated $\textbf{T}[x]$ as $D_{h + 1}$. Let now
$v_{\ell}$, $\ldots$, $v_{1}$ be the path in the lattice-based Merkle tree $T$ from node $v_{\ell}$ ($v_{\ell}$ stores the value  $\textbf{T}[x]$) to the child $v_{1}$ of the root $r$ of $T$.
\end{quote}


\begin{enumerate}
\item How is  $\ell$ calculated in our scenario?

\item What is meant by ``$v_{\ell}$ stores the value $\textbf{T}[x]$''? Is $v_{\ell}$ the leaf node of the Merkle tree?  If so, equation 5.5 would update a label for $v_{\ell}$, but that contradicts our understanding of the $initialize$ algorithm which assigns labels for only internal nodes.


\item Looking at the tree in Figure 7.1, we think the path is ($v_{22}$, $v_{11}$) where $\ell = 2$, because these are the only internal nodes for which there is a label to update.  (The root has a digest, $d_{h}$, and the leaf node has an entry in $\textbf{T}$.)  Additionally, the algorithm states the path is from $v_{\ell}$ to ``the child $v_{1}$ of the root $r$ of $T$" suggesting that the last element of the path in question is the child of the root node, not the root node itself. Is this correct?

\item A notation question about the statement ``outputting the updated $\textbf{T}[x]$ as $D_{h + 1}$".  Isn't \textbf{T} the whole table, whereas $\textbf{T}[x]$ is an element in \textbf{T}.  If so, is there not a type mismatch in that statement, where $D_{h + 1}$ is the whole data structure and $\textbf{T}[x]$ is just one element?

\item When/where does the root digest get updated (especially if our understanding of the path in the questions 2 and 3 is correct, i.e. the root is not included in that path)?  Shouldn't there be a similar update for the root or the root digest?

\end{enumerate}

\section{Matrix Multiplication in Ruby}


\subsection{The \texttt{[row $\times$ column]} multiplication:}
\begin{itemize}
\item $ C = A_{n \times n} \times B_{n \times n}$
\item where $ c_{ij} = \sum\limits_{k=1}^n a_{ik} . b_{kj}   $
\item complexity : $\Theta(n^3)$ \\
\end{itemize}


\subsection{Divide and Conquer}

\begin{itemize}

\item Each of the matrices is divided into four $\frac{n}{2} \times \frac{n}{2}$
\item The multiplication of two $ n \times  n $ matrices is reduced to 8 multiplications of
two $\frac{n}{2} \times \frac{n}{2}$ matrices and four $\frac{n}{2} \times \frac{n}{2}$ matrix additions.
\item recurrence : $T(n) = 8.T. \frac{n}{2} + \Theta(n^2) \rightarrow \Theta(n^3)$ \\
\end{itemize}

\subsection{Strassen's Algorithm}

\begin{itemize}

\item Bulky for implementation and not numerically stable
\item Useful and advantageous for large matrices
\item $ \Theta(n^{log7}) \approx \Theta(n^{2.807})$ \\
\end{itemize}

\subsection{Coppersmith Winograd Algorithm}

\begin{itemize}

\item Theoretical optimum; never used in practice.
\item $ \Theta(n^{2.376}) $ \\
\end{itemize}

\end{document}